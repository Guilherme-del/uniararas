- Criar um programa que aplique um ruído aleatório no sinal a baixo, e através de FFT identifique o ruído aplicado.
- Em seguida pesquise sobre filtros que possam ser aplicados para corrigir o ruído, e desenvolva a solução.
a nota de MA02 será a entrega da pesquisa sobre os filtros, e o plot dos gráficos do sinal que deverão ser gerados pelo programa criado. Ou, seja seu programa deverá plotar o gráfico do sinal ruidoso, posteriormente FFT com a identificação do sinal e do ruído, e finalmente a resposta do sinal após a aplicação do filtro.

-P02 será a entrega dos códigos fonte do programa criado.

Data da Entrega. 
13/06/2023 até as 22h


***SINAL***

% Amostragem
fs = 200; % Freq de amostragem em Hz
Ts = 1/fs; % Período de amostragem em s
f = 10; % Freq. do sinal em Hz
T=1/f; % Período do sinal em s
t = -3*T:Ts:3*T; % vetor de tempo com 3 períodos antes e depois de t=0

% Sinal senoidal perfeito (sem ruído)
sinal = 10*sin(2*pi*f*t);


******************************************************************
EXEMPLO EM OCTAVE PARA GERAR O SINAL RUIDOSO E PLOT.

clc
clear all
close all

pkg load signal

% Amostragem
fs = 200; % Freq de amostragem em Hz
Ts = 1/fs; % Período de amostragem em s
f = 10; % Freq. do sinal em Hz
T=1/f; % Período do sinal em s
t = -3*T:Ts:3*T; % vetor de tempo com 3 períodos antes e depois de t=0

% Sinal senoidal perfeito (sem ruído)
sinal = 10*sin(2*pi*f*t);


%fruido = 50;
%ruido = 2*cos(2*pi*fruido*t);
% Ruído aleatório - simulando erros de quantização
% de conversores analógico-digital (A/D)
% OBS: Altere a constante que multiplica o ruído para alterar a sua
% amplitude
ruido = randn(1,length(sinal))*2;
SinalRuidoso = sinal+ruido;

subplot(211)
plot(t,SinalRuidoso,'r','Linewidth',2)
xlabel('Tempo (s)')
ylabel('Amplitude')
title('Sinal ruidoso')
set(gca,'Fontsize',16)