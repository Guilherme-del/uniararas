\documentclass[conference]{IEEEtran}
\usepackage[utf8]{inputenc}
\usepackage{listings}
\usepackage{graphicx}
\usepackage{hyperref}

\title{Uma Linguagem Específica de Domínio (DSL) para Resolução do Problema da Mochila em Python}

\author{
    \IEEEauthorblockN{Guilherme Cavenaghi - 109317,
    Rafael Godoy - 110453,
    Rafael Pereira de Souza - 109680}
    \IEEEauthorblockA{Centro Universitário FHO\\
    \{guilherme.cavenaghi, Rafagodoy330, r.souza\}@alunos.fho.edu.br}
}

\begin{document}

\maketitle

\begin{abstract}
Este trabalho apresenta o desenvolvimento de uma linguagem específica de domínio (DSL) voltada à resolução do Problema da Mochila (Knapsack Problem), utilizando ferramentas de análise léxica e sintática em Python. A proposta permite representar instâncias do problema em uma linguagem legível e interpretável, com posterior execução por uma máquina virtual baseada em algoritmos P e NP. O objetivo é estudar a complexidade computacional e aplicar conceitos de teoria da computação em diferentes linguagens formais.
\end{abstract}

\section{Introdução}
O Problema da Mochila 0/1 é um clássico problema de otimização combinatória e pertence à classe NP-completo \cite{martello1990knapsack}. Sua importância teórica e prática motivou a criação de ferramentas de modelagem e resolução. Neste contexto, propomos uma linguagem específica de domínio (DSL) que permite ao usuário descrever instâncias do problema de forma simples, sendo interpretadas por um analisador léxico, parser e máquina virtual escritos em Python.

\section{Análise do Artigo-Base}
Inspiramo-nos em trabalhos que utilizam DSLs para modelagem de problemas em logística e produção, adaptando o conceito à modelagem matemática do Knapsack Problem. As abordagens anteriores focam em sistemas WMS, enquanto este trabalho foca em complexidade computacional e algoritmos.

\section{Definição do Novo Contexto}
O problema da mochila é ideal para comparar a aplicabilidade de algoritmos das classes P e NP \cite{cormen2022algorithms}. Criar uma linguagem específica para esse fim permite estudar linguagens formais, gramáticas, autômatos e análise sintática em um contexto prático de TCC e teoria da computação.

\section{Adaptação da Abordagem Metodológica}
Utilizamos a biblioteca PLY (Python Lex-Yacc) para implementar o lexer e o parser. A gramática foi projetada em EBNF para permitir a entrada de instâncias da mochila com lista de itens, pesos, valores e capacidade. O uso de uma DSL para esse tipo de problema segue recomendações de projetos similares apresentados em \cite{mernik2005dsl}.

\subsection{Exemplo de Entrada da DSL}
\begin{lstlisting}
mochila capacidade=50:
  item "Notebook" peso=20 valor=2000
  item "Livro" peso=5 valor=300
  item "Fone" peso=1 valor=500
\end{lstlisting}

\section{Desenvolvimento da Aplicação}
\begin{itemize}
    \item Linguagem: Python 3.11
    \item Bibliotecas: PLY para lexer/parser, argparse para CLI
    \item Etapas:
    \begin{enumerate}
        \item Definição da gramática da DSL
        \item Implementação do lexer
        \item Implementação do parser
        \item Criação de interpretador e resolução do problema com algoritmos exato (NP) e heurístico (P)
    \end{enumerate}
\end{itemize}

\section{Justificativas das Escolhas Técnicas}
Python foi escolhido pela facilidade de prototipagem e por possuir bibliotecas robustas para construção de compiladores. A representação DSL textual permite reusabilidade, clareza e padronização das instâncias.

\section{Resultados}
Resultados preliminares mostram que o tempo de entrada de dados via DSL é reduzido, e a resolução com heurística apresenta desempenho satisfatório frente a soluções exatas com instâncias pequenas e médias.

\section{Conclusão}
A implementação da DSL permitiu explorar de maneira prática conceitos da teoria da computação, em especial linguagens formais e classes de complexidade. O projeto é extensível para resolução de outros problemas NP-Completo.

\section*{Repositório}
Disponível em: \href{https://github.com/Guilherme-del/uniararas/tree/master/python/teoriaCompiladores/n2}{GitHub}

\bibliographystyle{IEEEtran}
\bibliography{referencias}

\end{document}
