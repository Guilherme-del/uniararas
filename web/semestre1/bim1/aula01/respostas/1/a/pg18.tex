Se em um mundo hipotético não existisse a W3C,
teríamos a web universal pautada na acessibilidade?

R: É improvável que teríamos a web universal pautada na acessibilidade se não existisse a W3C
Os padrões desenvolvidos pela W3C, como o HTML, CSS, e o WCAG, 
estabelecem regras claras para a criação de conteúdo web acessível e garantem que os desenvolvedores 
sigam as melhores práticas para tornar o conteúdo da web acessível a todos os usuários.


Qual motivo teria um web designer para utilizar RIA(Rich Internet Application)?
R: Melhor experiência do usuário,Funcionalidades avançadas,Melhor desempenho,Facilidade de manutenção e 
Flexibilidade.