Quais são as vantagens, desvantagens e necessidades de uso para CSS:
R:
Vantagens:
  -Controle de design: o CSS permite que os desenvolvedores controlem a aparência de uma página da web, separando a apresentação do conteúdo.
  -Facilidade de uso: o CSS é relativamente fácil de aprender, especialmente para aqueles que já têm experiência em HTML.
  -Eficiência: o CSS permite que você defina estilos uma vez e aplique-os a vários elementos em uma página da web, tornando a manutenção e atualização de uma página da web mais eficiente.
Desvantagens:
  -Problemas de compatibilidade: nem todos os navegadores da web suportam todas as funcionalidades do CSS, o que pode levar a diferenças na aparência da página da web em diferentes navegadores.
  -Complexidade: embora o CSS seja relativamente fácil de aprender, pode se tornar complexo ao criar estilos para páginas da web mais avançadas.
  -Demanda por habilidades de design: a criação de estilos visualmente atraentes pode exigir habilidades de design gráfico, o que pode ser um desafio para desenvolvedores sem essa experiência.
Necessidades de uso:
  -Criação de páginas da web: o CSS é essencial para definir a aparência e o layout de páginas da web.
  -Integração com outras tecnologias: o CSS é frequentemente combinado com HTML e JavaScript para criar páginas da web interativas e visualmente atraentes, portanto, o conhecimento dessas tecnologias é essencial para qualquer desenvolvedor web.
  -Adaptabilidade: o CSS é uma tecnologia em constante evolução e, portanto, é importante para os desenvolvedores manterem-se atualizados com as novas funcionalidades e práticas recomendadas.