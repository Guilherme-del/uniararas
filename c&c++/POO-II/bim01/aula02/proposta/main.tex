Crie uma classe denominada Elevador para armazenar as informações de um elevador dentro de um prédio.
A classe deve armazenar:
    Andar atual (0=térreo),
    total de andares no prédio, excluindo o térreo
    capacidade do elevador,
    quantas pessoas estão presentes nele.

    A classe deve também disponibilizar os seguintes métodos:

    inicializa: que deve receber como parâmetros: a capacidade do elevador e o total de andares no prédio
                     (os elevadores sempre começam no térreo e vazios);
    entra:       para acrescentar uma pessoa no elevador (só deve acrescentar se ainda houver espaço);
    sai:           para remover uma pessoa do elevador (só deve remover se houver alguém dentro dele);
    sobe:       para subir um andar (não deve subir se já estiver no último andar);
    desce:     para descer um andar (não deve descer se já estiver no térreo);
    get....:      métodos para obter cada um dos os dados armazenados.